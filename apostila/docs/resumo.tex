%%%%%%%%%%%%%%%%%%%%%%%%%%%%%%%%%%%%%%%%%%%%%%%%%%%%%%%%%%%%%%%%%%%%%%%%%%%%%%%%
% RESUMO %% obrigatório

\begin{resumo}

%% neste arquivo resumo.tex
%% o texto do resumo e as palavras-chave têm que ser em Português para os documentos escritos em Português
%% o texto do resumo e as palavras-chave têm que ser em Inglês para os documentos escritos em Inglês
%% os nomes dos comandos \begin{resumo}, \end{resumo}, \palavraschave e \palavrachave não devem ser alterados

\hypertarget{estilo:resumo}{} %% uso para este Guia

%Neste trabalho é analisada a possível natureza caótica da turbulência atmosférica. As análises aqui realizadas, baseadas em dados de temperatura de alta resolução, obtidos pela campanha WETAMC do projeto LBA, sugerem a existência de um comportamento caótico de baixa dimensão na camada limite atmosférica. O atrator caótico correspondente possui uma dimensão de correlação de $D_{2}=3.50\pm0.05$. A presença de dinâmica caótica nos dados analisados é confirmada com a estimativa de um expoente de Lyapunov pequeno mas positivo, com valor $\lambda_{1}=0.050\pm0.002$. No entanto, esta dinâmica caótica de baixa dimensão está associada à presença das estruturas coerentes na camada limite atmosférica e não à turbulência atmosférica. Esta afirmação é evidenciada pelo processo de filtragem por wavelets utilizado nos dados experimentais estudados, que permite separar a contribuição da estruturas coerentes do sinal turbulento de fundo.

Este material apresenta a linguagem de marcação \LaTeX{} para a confecção de textos científicos, tendo como foco o estilo de publicações do INPE. São apresentados os aspectos históricos de formulação da linguagem e as motivações para o seu uso dentro do ambiente acadêmico. O objetivo principal do uso da linguagem é permitir que o usuário concentre-se na escrita do texto, no desenvolvimento das suas ideias sem ter que se preocupar com a determinação e o posicionamento dos diversos elementos estruturais de um documento. Não se trata, porém, de um curso de escrita científica, mas sim de um manual objetivo para a aplicação da linguagem de marcação \LaTeX{} utilizando, especialmente, o estilo de publicações do INPE. Ao consultar o conteúdo deste material, o usuário estará habilitado a aplicar a linguagem na elaboração e desenvolvimento dos seus trabalhos acadêmicos e científicos.

\palavraschave{%
	\palavrachave{\LaTeX{}}%
	\palavrachave{Escrita Científica}%
	\palavrachave{Linguagem de Marcação}%
}
 
\end{resumo}