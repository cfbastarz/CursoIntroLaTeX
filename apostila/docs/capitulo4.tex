\chapter{Parte IV - Apresentações e Pôsteres}
\label{cap:parteIV}

\section{Pacote Beamer}
\label{sec:beamer}

O Beamer é o pacote padrão do LaTeX para a produção de apresentações no estilo do Microsoft PowerPoint. Assim como os documentos do LaTeX, é possível reconhecer os documentos de apresentações produzidos pelo Beamer pela sua qualidade gráfica e estilos pré-definidos (é possível criar estilos a partir do zero, mas esta tarefa não será abordada aqui).

\subsection{Estilos}
\label{sec:estilos}

\subsection{Transições e Animações}
\label{sec:trans_anima}

\subsection{Elementos dos Slides}
\label{sec:elem_slides}

\section{Pacote TColorBox}
\label{sec:tcolorbox}

O pacote TColorBox é um pacote muito interessante de ser utilizado. Os elementos do texto que contém dicas, comandos, exemplos e exercícios que se encontram neste documento, foram todos produzidos utilizando este pacote. Com ele é possível também produzir pôsteres em qualquer formato.

%\section{Confecção de Pôsteres e Apresentações}
%\subsection{Estilos, equações, gráficos e tabelas}

%\section{Apresentações}
%\subsection{Construção de figuras e diagramas, pequenas animações e configurações especiais}

%\section{Pôsteres}
%\subsection{Definindo estilos}

%\begin{dica}{Dica \#n}
%É possível também utilizar o pacote \mintinline{latex}{tcolorbox} para fazer pôsteres também!
%\end{dica}
%\begin{marker}
%É possível também utilizar o pacote %\mintinline{latex}{tcolorbox} para fazer pôsteres também!
%\end{marker}

% https://tex.stackexchange.com/questions/486009/tcolorboxes-side-by-side