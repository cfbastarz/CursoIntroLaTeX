% Apêndice B
\hypertarget{estilo:apendice1}{} %% uso para este Guia

\renewcommand{\thechapter}{}%

\chapter{APÊNDICE B - OPÇÔES DE COMPILAÇÂO} % trocar A por B na próxima apêndice e etc
\label{apendiceB} % trocar A por B na próxima apêndice e etc

\renewcommand{\thechapter}{B} % trocar A por B na próxima apêndice e etc

Neste apêndice são apresentadas algumas opções úteis para a compilação de documentos \LaTeX{}. No Capítulo \ref{cap:parteI}, foi mostrado como um documento \LaTeX{} simples pode ser compilado utilizando-se os seguintes comandos:

\begin{meucomando}
latex arquivo.tex 
dvips arquivo.dvi
ps2pdf arquivo.ps
\end{meucomando}

Na sequência de comandos acima, até que o resultado final seja alcançado (o arquivo PDF), observe que há vários arquivos intermediários que são gerados durante a compilação. O comando {\tt latex arquivo.tex} irá gerar os seguintes arquivos:

\begin{itemize}
  \item {\tt arquivo.aux}: contém as informações sobre a estrutura e as partes do documento (e.g., sumário, capítulos, seções etc);
  \item {\tt arquivo.dvi}: é o arquivo que contém o documento formatado, pronta para inspeção no computador (ainda não é o arquivo final);
  \item {\tt arquivo.log}: contém informações o compilador utilizado e a versão, o tipo de classe empregada, estilos e pacotes carregados.
\end{itemize}

O passo seguinte à execução do comando {\tt latex}, é criar um arquivo que possa ser impresso. Esse arquivo pode ser um arquivo PS, EPS ou PDF, sendo o arquivo PDF o formato preferível. Para criar o arquivo no formato PS, utiliza-se o comando:

\begin{meucomando}
dvips arquivo.dvi 
\end{meucomando}

Para criar o arquivo PDF, há duas formas: converter o arquivo DVI em PDF (sem a necessidade do arquivo PS), a partir do comando: 

\begin{meucomando}
dvipdf arquivo.dvi 
\end{meucomando}

A outra forma, é converter o arquivo PS em PDF, utilizando o comando a seguir:

\begin{meucomando}
ps2pdf arquivo.ps 
\end{meucomando}

Entretando, ao invés de utilizar a sequência de comandos acima, pode-se utilizar o comando {\tt pdflatex} que automatiza as etapas intermediárias entre o documento \LaTeX{} e o arquivo PDF final:

\begin{meucomando}
pdflatex arquivo.tex
\end{meucomando}

Outros compiladores, como o \XeLaTeX{} escrevem o arquivo PDF de forma direta, bastando, para isso, utilizar o seguinte comando:

\begin{meucomando}
xelatex arquivo.tex
\end{meucomando}

Dependendo das necessidades do usuário ou da complexidade do documento escrito (como esta apostila), pode ser necessário utilizar algumas opções do compilador para que o documento seja criado. O comportamento padrão dos compiladores do \LaTeX{}, é abortar o processo de compilação quando erros são encontrados. Por exemplo, quando um comando é digitado errado, como indicado a seguir:

\begin{meucomando}
latex arquivo.tex
This is pdfTeX, Version 3.14159265-2.6-1.40.20 (TeX Live 2019) (preloaded format=latex)
 restricted \write18 enabled.
entering extended mode
(./arquivo.tex
LaTeX2e <2019-10-01> patch level 3
(/usr/local/texlive/2019/texmf-dist/tex/latex/base/article.cls
Document Class: article 2019/10/25 v1.4k Standard LaTeX document class
(/usr/local/texlive/2019/texmf-dist/tex/latex/base/size10.clo))
(/usr/local/texlive/2019/texmf-dist/tex/latex/base/inputenc.sty)
(/usr/local/texlive/2019/texmf-dist/tex/latex/lipsum/lipsum.sty
(/usr/local/texlive/2019/texmf-dist/tex/latex/l3kernel/expl3.sty
(/usr/local/texlive/2019/texmf-dist/tex/latex/l3kernel/expl3-code.tex
...
(/usr/local/texlive/2019/texmf-dist/tex/latex/lipsum/lipsum.ltd.tex))
No file arquivo.aux.
! Undefined control sequence.
l.14 \maketitlee

?
\end{meucomando}

Na saída do comando acima, observe que o compilador não entendeu a instrução \mintinline{latex}{\maketitlee} e parou a compilação com a mensagem {\tt ! Undefined control sequence.}. Neste caso, o compilador aguarda algum tipo de instrução, o que é conveniente pois permite que o usuário possa identificar e corrigir os problemas. Neste caso, para sair do \textit{prompt} de compilação do \LaTeX{}, basta digitar a palavra \textit{quit}. Veja a seguir:

\begin{meucomando}
...
No file arquivo.aux.
! Undefined control sequence.
l.14 \maketitlee

? quit
OK, entering \batchmode
\end{meucomando}

Quando algum pacote desatualizado, ou que causa algum tipo de conflito com outro pacote ocorre, ou quando comandos errados são utiizados (e o usuário não percebe), a opção {\tt -interaction=nonstopmode} pode ser útil. Esta opção instrui o compilador em uso a continuar com a compilação do documento, mesmo que erros sejam encontrados. Ela deve ser utilizada da seguinte forma: 

\begin{meucomando}
latex -interaction=nonstopmode arquivo.tex
This is pdfTeX, Version 3.14159265-2.6-1.40.20 (TeX Live 2019) (preloaded format=latex)
 restricted \write18 enabled.
entering extended mode
(./arquivo.tex
LaTeX2e <2019-10-01> patch level 3
(/usr/local/texlive/2019/texmf-dist/tex/latex/base/article.cls
Document Class: article 2019/10/25 v1.4k Standard LaTeX document class
(/usr/local/texlive/2019/texmf-dist/tex/latex/base/size10.clo))
(/usr/local/texlive/2019/texmf-dist/tex/latex/base/inputenc.sty)
(/usr/local/texlive/2019/texmf-dist/tex/latex/lipsum/lipsum.sty
(/usr/local/texlive/2019/texmf-dist/tex/latex/l3kernel/expl3.sty
(/usr/local/texlive/2019/texmf-dist/tex/latex/l3kernel/expl3-code.tex
...
(/usr/local/texlive/2019/texmf-dist/tex/latex/lipsum/lipsum.ltd.tex))
(./arquivo.aux)
! Undefined control sequence.
l.14 \maketitlee

[1] (./arquivo.aux) )
(see the transcript file for additional information)
Output written on arquivo.dvi (1 page, 2232 bytes).
Transcript written on arquivo.log.
\end{meucomando}

Note que mesmo com a instrução \mintinline{latex}{\maketitlee} estando incorreta (o correto é \mintinline{latex}{\maketitle}), o compilador continuou com a compilação até o fim. Neste caso, o resultado é um documento sem o título, nome do autor e data.

Alguns pacotes, requerem opções especiais para que possam ser utilizados. Um exemplo é o pacote {\tt tcolorbox} quando é utilizado para produzir listagem de código (como a maioria dos exemplos mostrados nesta apostila). Neste caso, faz-se necessária a utilização do pacote {\tt Pygments} do \textit{Python}. Como este pacote não pertence ao \LaTeX{}, é necessário utilizar a opção {\tt -shell-escape} para que o pacote {\tt tcolorbox} possa utilizar o pacote {\tt Pygments}. Neste caso, a utilização desta ocorre da seguinte forma:

\begin{meucomando}
latex -shell-escape arquivo.tex
\end{meucomando}

Alguns editores tem a funcionalidade de atualizar automaticamente o documento final quando o código-fonte é salvo, como é o caso do editor \textit{online} \textit{Overleaf}. Quando o \LaTeX{} é utilizado em linha de comando, é possível também obter um comportamente semelhante utilizando o comando {\tt latexmk}: enquando o documento é editado e salvo, utilizando um editor local como o VIM, pode-se deixar o comando a seguir sendo executado em um outro terminal (no mesmo diretório):

\begin{meucomando}
latexmk -pvc -pdf -e '$pdflatex=q/xelatex %O -interaction=nonstopmode %S/' arquivo.tex 
\end{meucomando}   

Após executar este comando (o terminal ficará ocupado), ocorrerá o seguinte:

\begin{meucomando}
------------
Run number 1 of rule 'pdflatex'
------------
------------
Running 'xelatex  -recorder  -interaction=nonstopmode "arquivo.tex"'
------------
This is XeTeX, Version 3.14159265-2.6-0.999991 (TeX Live 2019) (preloaded format=xelatex)
 restricted \write18 enabled.
entering extended mode
(./arquivo.tex
LaTeX2e <2019-10-01> patch level 3
(/usr/local/texlive/2019/texmf-dist/tex/latex/base/article.cls
Document Class: article 2019/10/25 v1.4k Standard LaTeX document class
(/usr/local/texlive/2019/texmf-dist/tex/latex/base/size10.clo))
(/usr/local/texlive/2019/texmf-dist/tex/latex/base/inputenc.sty

Package inputenc Warning: inputenc package ignored with utf8 based engines.

(/usr/local/texlive/2019/texmf-dist/tex/latex/lipsum/lipsum.sty
(/usr/local/texlive/2019/texmf-dist/tex/latex/l3kernel/expl3.sty
(/usr/local/texlive/2019/texmf-dist/tex/latex/l3kernel/expl3-code.tex
...
(/usr/local/texlive/2019/texmf-dist/tex/latex/lipsum/lipsum.ltd.tex))
(./arquivo.aux) [1] (./arquivo.aux) )
Output written on arquivo.pdf (1 page).
Transcript written on arquivo.log.
=== TeX engine is 'XeTeX'
Latexmk: Log file says output to 'arquivo.pdf'
Latexmk: All targets () are up-to-date
Latexmk: I have not found a previewer that is already running.
   So I will start it for 'arquivo.pdf'
------------
For rule 'view', running '&if_source(  )' ...
------------
Running 'start open "arquivo.pdf"'
------------

=== Watching for updated files. Use ctrl/C to stop ...
\end{meucomando}

Observe a mensagem {\tt === Watching for updated files. Use ctrl/C to stop ...}. Se o documento PDF estiver aberto em um visualizador como o \textit{Adobe Acrobat} ou o \textit{Evince}, o documento deverá ser atualizado sempre que o código-fonte for salvo.

Por fim, todo o conteúdo desta apostila pode ser acessado a partir de \url{https://github.com/cfbastarz/CursoIntroLaTeX}. Para obter uma cópia do repositório, basta seguir os passos abaixo:

\begin{meucomando}
git clone git@github.com:cfbastarz/CursoIntroLaTeX.git
\end{meucomando}

Para compilar o código, será necessário ter o pacote {\tt Pygments} do \textit{Python} instalado na máquina. Para instalar este pacote, basta utilizar um dos comandos a seguir:

\begin{meucomando}
sudo easy_install Pygments
\end{meucomando}

ou

\begin{meucomando}
sudo apt-get install python-pygments
\end{meucomando}

A apostila (e os outros materiais), podem ser finalmente compilados utilizando o compilador \XeLaTeX{} e o Bib\TeX{}:

\begin{meucomando}
xelatex -interaction=nonstopmode -shell-escape publicacao.tex
bibtex publicacao
xelatex -interaction=nonstopmode publicacao.tex
xelatex -interaction=nonstopmode publicacao.tex
\end{meucomando}
