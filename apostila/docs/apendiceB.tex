%%%%%%%%%%%%%%%%%%%%%%%%%%%%%%%%%%%%%%%%%%%%%%%%%%%%%%
%Apêndice A
\hypertarget{estilo:apendice1}{} %% uso para este Guia
%Este apêndice foi criado apenas para indicar como construir um apêndice no estilo, não existia no original da tese.
%%%%%%%%%%%%%%%%%%%%%%%%%%%%%%%%%%%%%%%%%%%%%%%%%%%%%%
\renewcommand{\thechapter}{}%
\chapter{APÊNDICE B - OPÇÔES AVANÇADAS DE COMPILAÇÂO}  % trocar A por B na próxima apêndice e etc
\label{apendiceB} % trocar A por B na próxima apêndice e etc
\renewcommand{\thechapter}{B}%    % trocar A por B na próxima apêndice e etc

Neste apêndice são apresentadas algumas opções úteis para a compilação de documentos \LaTeX{}. No Capítulo \ref{cap:parteI}, foi mostrado como um documento \LaTeX{} simples pode ser compilado utilizando-se a linha de comando:

\begin{meucomando}
latex exe_doc.tex 
dvips exe_doc
ps2pdf exe_doc
\end{meucomando}

De forma mais simples e direta, pode-se utilizar o comando {\tt pdflatex} que automatiza as etapas intermediárias entre o documento no formato {\tt .tex} e o arquivo PDF final:

\begin{meucomando}
pdflatex exe_doc.tex
\end{meucomando}

Porém, dependendo das necessidades do usuário ou da complexidade do documento escrito (como esta apostila), pode ser necessário utilizar algumas opções mais avançadas do compilador. Algumas opções mais comuns são...

% xelatex -interaction=nonstopmode -shell-escape publicacao.tex
% biblatex publicacao
% xelatex -interaction=nonstopmode publicacao.tex
% xelatex -interaction=nonstopmode publicacao.tex

% Utilizando o latexmk
%  latexmk -xelatex -interaction=nonstopmode -shell-escape -bibtex publicacao.tex

% A opção a seguir pode ser utilizada para emular o comportamento de alguns editores, em que o documento é automaticamente atualizado quando ele é salvo:
% latexmk -pvc -pdf -e '$pdflatex=q/xelatex %O -interaction=nonstopmode %S/' publicacao.tex 
