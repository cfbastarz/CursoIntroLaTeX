\chapter{Parte I - Preparação}
\label{cap:parteI}

\section{Introdução}
\label{sec:intro}

Metodologia científica compreende os conhecimentos necessários para a produção científica. Artigos, relatórios, dissertações e teses são documentos que trazem relatos de experimentações, muitas vezes práticas, que surgem da necessidade de se testar hipóteses. Estas hipóteses frequentemente se referem ao mundo físico em vivemos, mas podem também, serem formulados a cerca de ideias abstratas. O método científico, assim como preconizou o patrono de todas as ciências, Renè Descartes, em seu ``O Discurso do Método'' \cite{descartes}, representa uma sequência de etapas que visam testar as hipóteses que formulamos e transformá-las, então, em teses, em teorias.

A escrita é parte fundamental da metodologia científica. É através dela que documentamos todo o processo de desenvolvimento da ciência, é através dela que se faz a comunicação formal da ciência que se produz e que, finalmente, se materializa o conhecimento adquirido. A escrita científica deve ser pautada por normas que ajudam a verificar a natureza do que se escreve e a validade dos argumentos com que se trata o objeto de estudo.

O \TeX{} (pronuncia-se ``Tec'', sendo o ``X'' ao final a letra $\chi$ do alfabeto Grego), foi criado pelo Matemático e Cientista da Computação americano Donald Ervin Knuth, em 1978 para facilitar a escrita e melhorar a apresentação de textos científicos, principalmente aqueles com notações matemáticas. Naquela época, não haviam editores de texto formatado, como por exemplo o \textit{Microsoft Word}, \textit{Libre Office} e outros. Estes software viriam a ser lançados apenas na década de 1990. Além disso, o computador pessoal estava em seus primórdios, e viria a se tornar realmente pessoal com o lançamento do Apple II, no final dos anos 1970. Este tipo de computador, incluindo os seus clones (i.e., demais computadores que possuíam o mesmo \textit{layout} de processador e memória de acesso randômico) não possuíam interfaces gráficas, mouses e discos rígidos; eles possuíam apenas um compilador BASIC e tela monocromática. Tendo-se em vista este cenário, a produção científica já havia avançado, pois o computador sempre foi uma ferramente essencial nas mais diversas áreas do conhecimento (imagine como se comunicava a ciência antes do advento dos computadores, ou mesmo antes da invenção da prensa!). 

Por outro lado, com os avanços tecnológicos e a sofisticação dos computadores pessoais, houve também a necessidade de se melhorar a representação tipográfica dos textos científicos, além da qualidade gráfica de imagens e gráficos. Em 1986, Leslie Lamport lança a primeira versão do \LaTeX{}, uma versão aprimorada e de mais fácil utilização do que o \TeX{} puro. De forma geral, as linguagens de marcação de texto como o \TeX{} e o \LaTeX{}

É fato que a confecção de documentos utilizando o \LaTeX{} pode ser um pouco trabalhosa, visto que a linguagem é focada na escrita, e não na formatação como é o caso das suítes de escritório como o \textit{Microsoft Office}. Neste caso, o usuário deve ponderar sobre o tipo de documento que tem por intenção produzir e o tamanho que este virá a ter. Portanto, a escrita de um documento utilizando o \LaTeX{} é vantajosa quando: 1) um estilo já está preparado (e.g., um artigo científico, dissertação, tese, relatório etc); 2) quando muitos elementos textuais estiverem presentes (e.g., figuras, tabelas, referências cruzadas, quebras de seções etc); 3) quando se tem tempo suficiente. Entre estes três pontos, há que se ressaltar o tempo necessário para a escrita de documentos utilizando \LaTeX{}. Embora a linguagem permita que a escrita seja focada no conteúdo do texto (ao invés do seu aspecto), há uma curva de aprendizado e é bastante frequente o usuário da linguagem se encontrar em situações em que necessita criar uma tabela um pouco mais complicada, ou inserir um conjunto de figuras de uma forma diferente. Estas situações não são óbvias de serem resolvidas e o usuário precisa ou ler a documentação dos pacotes (o que acaba não sendo muito prático), ou recorre a Fóruns na internet para resolver seus problemas. Há vários fóruns na internet que são especializados na linguagem \LaTeX{} e são uma boa fonte para a solução de diversas dúvidas e problemas. Apesar disso, o efeito colateral disso é que o usuário acaba não aprendendo como usar efetivamente a linguagem, porque nunca está a par da documentação, tendo à mão apenas receitas prontas. Logo, esta apostila foi escrita como uma forma de orientar os usuários a adquirirem o mínimo de independência na utilização do \LaTeX{}.

% Explicar as diferenças entre o TeX e o LaTex.

O texto do documento está organizado em três partes: a ``Parte I - Entendendo o \LaTeX{}'' foi preparada como um introdução à linguagem de marcação \LaTeX{}, a fim de facilitar o entendimento das funções e estruturas básicas da linguagem, que serão extensamente utilizadas nos estilos; a ``Parte II - Estilos do INPE'', contém um manual de utilização dos estilos do INPE para a escrita de dissertações e teses. Se o leitor já possui alguma experiência com a linguagem de marcação \LaTeX{}, ela pode pular diretamente para esta parte. Na ``Parte III - Pacote Beamer'', há a apresentação básica do pacote Beamer, frequentemente utilizado para a confecção de apresentações (no estilo do \textit{Microsoft PowerPoint}) e pôsteres.

\section{Objetivos}
\label{sec:objetivos}

Nesta apostila são apresentados os conceitos fundamentais da linguagem de marcação \LaTeX{}, com especial atenção à utilização dos estilos do INPE para a escrita de dissertações e teses. Os objetivos específicos são:

\begin{itemize}
    \item Apresentar a linguagem de marcação \LaTeX{}, acompanhado de um breve histórico do seu desenvolvimento;
    \item Mostrar ao usuário como instalar o compilador/interpretador da linguagem nas plataformas mais conhecidas;
    \item Apresentar ao usuário os conceitos fundamentais da linguagem, levando-o a ter independência na utilização dos estilos do INPE;
    \item Treinar o usuário na utilização dos estilos de INPE para a escrita de dissertações e teses.
\end{itemize}

\section{Estrutura e Organização do Documento}
\label{sec:estrutura}

Este documento foi preparado utilizando o estilo de teses e dissertações do INPE, com a finalidade de servir não apenas como uma manual de utilização do estilo, mas também como um documento simples que possa ser utilizado como uma referência no aprendizado da linguagem de marcação LaTeX. Para cumprir com esta finalidade, ao longo das seções que se seguem, alguns elementos textuais foram incorporados para sinalizar instruções específicas, como comandos do ambiente \textit{Shell} do Linux e dicas ou instruções sobre pontos específicos do que está sendo apresentado. 

Dessa forma, dicas e observações são destacados da seguinte forma:

\begin{marker}
Isto é uma dica ou uma observação!
\end{marker}

De outra forma, comando que devem ser digitados em algum ambiente computacional específico, são destacados como:

\begin{commandshell}
echo ''Isto é um comando Shell do Linux/UNIX!''
\end{commandshell}

Exemplos da linguagem são apresentados em uma caixa, contendo a grafia dos comandos e o seu resultado ame anexo. Exercícios são apresentados de forma semelhante, mas com a diferença de que é apresentado um exemplo (eg., uma tabela) o qual o usuário deverá reproduzir em ambiente local ou online configurado para tal. As respostas dos exercícios são então apresentadas no Anexo \ref{anexoA}. Ao longo do texto, o leitor irá notar que na maioria dos exemplos que contém algum tipo de texto, aparece um texto prolixo. Este texto é gerado automaticamente com o auxílio de um pacote específico ({\tt lipsum}), e não faz referência á nenhum tipo de conhecimento ou conceitos. Em outros exemplo, a frase ``\textit{The quick brown fox jumps over the lazy dog}'' é utilizada. Esta frase é um pangrama, e contém todas as letras do alfabeto da língua inglesa. Ela é frequentemente utilizada como prova tipográfica, o que permite verificar a renderização de todas as letras e do seu espaçamento em uma única frase, de acordo com o tipo de fonte utilizada.

Neste documento boa parte dos exemplos e trechos de código foi obtido da \textit{internet} de outros tutoriais. Uma lista de sites em que os exemplos foram obtidos, pode ser encontrado no Anexo \ref{anexoC}.

O documento está organizado em 4 partes. A Parte \ref{cap:parteI} trata da introdução e objetivos deste documento e da linguagem LaTeX. A Parte \ref{cap:parteII} apresentam uma introdução aos elementos e marcadores principais da linguagem. Ao final desta parte, o usuário deverá ser capaz de produzir documentos LaTeX simples, utilizando as classes mais comuns e os elementos textuais mais frequentes. Na Parte \ref{cap:parteIII}, é apresentado o estilo de INPE para a escrita de teses e dissertações. Ao final desta parte, o usuário deverá ser capaz de utilizar o estilo do INPE para a escrita de sua tese ou dissertação. É importante salientar, entretanto, que a Parte \ref{cap:parteIII} requer o aprendizado do conteúdo da Parte \ref{cap:parteII}. A Parte \ref{cap:parteIV} apresenta o pacote Beamer, uma classe que pode ser utilizada para confeccionar apresentações digitais.

\section{Preparação do Ambiente}
\label{sec:prepara}

O \LaTeX{} é, essencialmente, um compilador/interpretador. Para a sua utilização, é necessário instalar ele no computador. Nas seções a seguir, é mostrado como instalar o \LaTeX{} nos sistemas operacionais nos sistemas Windows, Linux e Mac OS. A utilização da linguagem pode ser feita de diversas formas, em linha de comando ou utilizando editores de texto puro ou ainda editores mais avançados do tipo \textit{What You See Is What You Get} (WYSIWYG). A utilização da linguagem será vista nas seções mais adiante.
 
\subsection{Escolhendo e instalando o compilador}
\label{sec:compilador}

%FALTA: EXPLICAR AS DIFERENÇAS ENTRE LATEX, PDFLATEX, XELATEX E LUATEX

Nas próximas seções, será mostrado como instalar e configurar o compilador/interpretador da linguagem. 

\subsection*{Linux}
\label{sec:linux}

Nos sistemas GNU Linux, a instalação dos pacotes do \LaTeX{} é bastante simples, mas pode variar de acordo com a distribuição utilizada. Neste manual, serão abordadas as distribuições mais populares e que utilizam os sistemas de pacotes ``apt'' (Debian e derivados) e ``dnf'' (RedHat e derivados). A vantagem destes gerenciadores de pacotes está no fato de que eles resolvem automaticamente as dependências, i.e., eles são capazes de instalar outros pacotes que são necessários para o correto funcionamento do programa principal. Em outras distribuições o processo de instalação pode ser diferente ou mesmo envolvendo a instalação a partir dos códigos fontes dos pacotes. 

O site oficial do \LaTeX{} é o \url{https://www.latex-project.org/}. No Linux, a principal distribuição da linguagem é o pacote ``texlive'' (\url{https://www.tug.org/texlive/}). Para instalar o pacote no Debian, basta fazer:

%\begin{minted}
%[
%frame=lines,
%framesep=2mm,
%baselinestretch=1.5,
%bgcolor=solarized-base2,
%fontsize=\normalsize,
%linenos
%]
%{bash}
%    $ sudo apt install texlive-full
%\end{minted}

\begin{commandshell}
sudo apt install texlive-full
\end{commandshell}

%\begin{tcblisting}{title={Snapshot of the staging area},
%gitexample={The option'-a'automatically stages all tracked and modifiedfiles before the commit.\par
%This can be combined with the message option'-m'as seen in the third line.}}
%git commit
%git commit -a
%git commit -am'changes to my example'
%\end{tcblisting}

No RedHat, basta fazer:

%\begin{minted}
%[
%frame=lines,
%framesep=2mm,
%baselinestretch=1.5,
%bgcolor=solarized-base2,
%fontsize=\normalsize,
%linenos
%]
%{bash}
%    $ sudo dnf install texlive-scheme-full
%\end{minted}

\begin{commandshell}
sudo dnf install texlive-scheme-full
\end{commandshell}

%\begin{dica}{Dica \# 1}
%Mesmo instalando o pacote completo do ``texlive'', é possível que outros pacotes precisem ser instalados depois.
%\end{dica}

\begin{marker}
Mesmo instalando o pacote completo do ``texlive'', é possível que outros pacotes precisem ser instalados depois.
\end{marker}

\subsection*{Windows}
\label{sec:windows}

No sistema operacional \textit{Microsoft Windows}, a instalação do pacote texlive pode ser feita de forma convencional, através do instalador oficial da distribuição disponível em \url{http://mirror.ctan.org/systems/texlive/tlnet/install-tl-windows.exe} (a url indicada sempre aponta para o pacote mais recente). Após baixar o pacote, siga as instruções a seguir para completar a instalação.

\begin{marker}
Outras informações sobre a instalação do LaTeX no sistema operacional \textit{Windows}, podem ser encontradas no documento \href{http://mtc-m16d.sid.inpe.br/col/sid.inpe.br/mtc-m19@80/2010/03.24.15.12/doc/ambiente_latex_no_windows.pdf}{oficial} do Serviço de Informação e Documentação (SID) do INPE.
\end{marker}

\subsection*{MacOS}
\label{sec:macos}

No MacOS, a forma mais simples de instalar o pacote do texlive, é a partir do instalador disponível em \url{http://tug.org/cgi-bin/mactex-download/MacTeX.pkg} (da mesma forma, este url sempre aponta para o pacote mais recente). Se você está habituado a utilizar algum tipo de gerenciador de pacote no MacOS, e.g., o ``brew'', pode tentar também a instalação com os seguintes comandos:

%\begin{minted}
%[
%frame=lines,
%framesep=2mm,
%baselinestretch=1.5,
%bgcolor=solarized-base2,
%fontsize=\normalsize,
%linenos
%]
%{bash}
%    $ brew install caskroom/cask/brew-cask
%    $ brew cask install mactex
%    $ brew cask install texmaker
%\end{minted}

\begin{commandshell}
brew install caskroom/cask/brew-cask
brew cask install mactex
brew cask install texmaker
\end{commandshell}