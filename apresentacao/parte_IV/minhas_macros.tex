% Macros extras
\definecolor{InpePreto}{HTML}{515151}
\definecolor{InpeAzul}{HTML}{017DBD}
\definecolor{InpeLaranja}{HTML}{F8931D}

% Caixas das Dicas (pacote tcolorbox)
% Material Colors
\newtcolorbox{marker}[1][]{enhanced,
  before skip=10mm,after skip=10mm,
  boxrule=1.4pt,left=6mm,right=3mm,top=2mm,bottom=2mm,
  colback=MaterialAmberA100!75,
  colupper=black,
  colframe=InpePreto,
  underlay={%
    \path[fill=none] (frame.south west) rectangle node[InpePreto]{\huge\bfseries !} ([xshift=7mm]frame.north west);
    }}
  
% Caixas dos Exemplos (pacote tcolorbox)
% Material Colors
\tcbset{
    texexp/.style={
        colframe=InpeLaranja!90,
        colback=white,
        coltitle=black,
        fonttitle=\small\sffamily\bfseries,
        fontupper=\small, 
        fontlower=\small,
        arc=0pt, outer arc=0pt, box align=center, halign=center, valign=center},
        example/.style 2 args={texexp,
        title={\hspace{-4mm}Exemplo \thetcbcounter: #1},label={#2}},
    }
    \newtcblisting{texexp}[1]{texexp,#1}
    \newtcblisting[auto counter,number within=section]{texexptitled}[3][]{%
        example={#2}{#3},#1}
    \newtcolorbox[use counter from=texexptitled]{texexptitledspec}[3][]{%
        example={#2}{#3},#1}

% Caixas dos Exercícios e Soluções (pacote tcolorbox)  
% Material Colors
\NewTColorBox[auto counter,number within=chapter]{exercise}{m+O{}}{%
    enhanced,
    colframe=MaterialGreen900,
    colback=white,
    coltitle=white,
    fonttitle=\bfseries,
    underlay={\begin{tcbclipinterior}
        \shade[inner MaterialGreen900,outer color=white]
            (interior.north west) circle (2cm);
        \draw[help lines,step=5mm,MaterialGreen900,shift={(interior.north west)}]
            (interior.south west) grid (interior.north east);
        \end{tcbclipinterior}},
    title={Exercício~ \thetcbcounter:},
    label={exercise:#1},
    attach title to upper=\quad,
    after upper={\par\hfill\textcolor{white}%
        {\itshape Resposta na página~\pageref{solution:#1}}},
    lowerbox=ignored,
    savelowerto=respostas/exercise-\thetcbcounter.tex,
    record={\string\solution{#1}{respostas/exercise-\thetcbcounter.tex}},
    #2
}

% Esta é a lista de exercícios
\newtcolorbox[auto counter,number within=section,list inside=exam]{texercise}[4][]{%
    texercisestyle,
    listing file={respostas/texercise\thetcbcounter.tex},
    label={exe:#2},
    record={\string\processsol{respostas/texercise\thetcbcounter.tex}{#2}},
    title={Exercício \thetcbcounter\hfill\mdseries Resposta na página \pageref{sol:#2}},
    list text={\vspace{-1em}\hspace{1mm} #3 - resposta na página \pageref{sol:#2}}, #1)}
    
\tcbset{texercisestyle/.style={arc=0.5mm, colframe=MaterialGreen900,
    colback=white, coltitle=white,
    fonttitle=\small\sffamily\bfseries, fontupper=\small, fontlower=\small,
    listing options={style=tcblatex,texcsstyle=*\color{MaterialRed900}},
}}

% \usepackage{hyperref} % for phantomlabel
\newtcbinputlisting{\processsol}[2]{%
    texercisestyle,
    listing only,
    listing file={#1},
    phantomlabel={sol:#2},%
    title={Resposta do Exercício \ref{exe:#2} na página \pageref{exe:#2}},
}

% Caixa dos Comandos (pacote tcolorbox)
\newtcblisting{meucomando}{listing engine=minted,
	minted style=colorful,
	minted language=bash,
	minted options={fontsize=\small,breaklines,autogobble,linenos,numbersep=3mm},
	colback=blue!5!white,colupper=black,colframe=InpePreto,listing only,
	left=5mm,enhanced,
	overlay={\begin{tcbclipinterior}\fill[red!20!blue!20!white] (frame.south west)
			rectangle ([xshift=5mm]frame.north west);\end{tcbclipinterior}}}

\newtcblisting{meucomandol}[1]{
	colback=blue!5!white,colupper=InpePreto,coltitle=black,colframe=InpeLaranja!90,listing only,
  title={\hspace{-5mm}#1},
  fonttitle=\small\sffamily\bfseries,
  arc=0pt, outer arc=0pt, box align=center, halign=center, valign=center, center,
  listing engine=minted,
	minted style=colorful,
	minted language=latex,
	minted options={fontsize=\small,breaklines,autogobble,linenos,numbersep=3mm},
	left=5mm,enhanced,
  overlay={ \begin{tcbclipinterior}\fill[red!20!blue!20!white] (frame.south west) rectangle ([xshift=5mm]frame.north west); \end{tcbclipinterior} }
  }

\newtcblisting{meucomandolf}[1]{
	colback=blue!5!white,colupper=InpePreto,coltitle=black,colframe=InpeLaranja!90,listing only,
  title={\hspace{-5mm}#1},
  fonttitle=\small\sffamily\bfseries,
  arc=0pt, outer arc=0pt, box align=center, halign=center, valign=center, center,
  width=0.99\paperwidth, height=0.99\paperheight,
  listing engine=minted,
	minted style=colorful,
	minted language=latex,
	minted options={fontsize=\small,breaklines,autogobble,linenos,numbersep=3mm},
	left=5mm,enhanced,
  overlay={ \begin{tcbclipinterior}\fill[red!20!blue!20!white] (frame.south west) rectangle ([xshift=5mm]frame.north west); \end{tcbclipinterior} }
  }

\newtcblisting{meucomandot}[1]{
	colback=blue!5!white,colupper=InpePreto,coltitle=black,colframe=InpeLaranja!90,listing only,
  title={\hspace{-5mm}#1},
  fonttitle=\small\sffamily\bfseries,
  arc=0pt, outer arc=0pt, box align=center, halign=center, valign=center, center,
  listing engine=minted,
	minted style=colorful,
	minted language=bash,
	minted options={fontsize=\small,breaklines,autogobble,linenos,numbersep=3mm},
	left=5mm,enhanced,
  overlay={ \begin{tcbclipinterior}\fill[red!20!blue!20!white] (frame.south west) rectangle ([xshift=5mm]frame.north west); \end{tcbclipinterior} }
  }
        
\newtcblisting{meucomandotf}[1]{
	colback=blue!5!white,colupper=InpePreto,coltitle=black,colframe=InpeLaranja!90,listing only,
  title={\hspace{-5mm}#1},
  fonttitle=\small\sffamily\bfseries,
  arc=0pt, outer arc=0pt, box align=center, halign=center, valign=center, center,
  width=0.99\paperwidth, height=0.99\paperheight,
  listing engine=minted,
	minted style=colorful,
	minted language=bash,
	minted options={fontsize=\small,breaklines,autogobble,linenos,numbersep=3mm},
	left=5mm,enhanced,
  overlay={ \begin{tcbclipinterior}\fill[red!20!blue!20!white] (frame.south west) rectangle ([xshift=5mm]frame.north west); \end{tcbclipinterior} }
  }

\DeclareTCBInputListing[]{\mylisting}{ m }{% m é um argumento mandatório (veja a documentação do pacote xparse)
  colback=blue!5!white,colupper=InpePreto,coltitle=black,colframe=InpeLaranja!90,listing only,
  title={\hspace{-5mm}#1},
  fonttitle=\small\sffamily\bfseries,
  arc=0pt, outer arc=0pt, box align=center, halign=center, valign=center, center,
  listing engine=minted,
	minted style=colorful,
  minted language=latex,
	minted options={fontsize=\small,breaklines,autogobble,linenos,numbersep=3mm},
	left=5mm,enhanced,
  } 

\DeclareTCBInputListing[]{\mylistingf}{ m }{% m é um argumento mandatório (veja a documentação do pacote xparse)
  colback=blue!5!white,colupper=InpePreto,coltitle=black,colframe=InpeLaranja!90,listing only,
  title={\hspace{-5mm}#1},
  fonttitle=\small\sffamily\bfseries,
  arc=0pt, outer arc=0pt, box align=center, halign=center, valign=center, center,
  width=0.99\paperwidth, height=0.99\paperheight,
  listing engine=minted,
	minted style=colorful,
  minted language=latex,
	minted options={fontsize=\small,breaklines,autogobble,linenos,numbersep=3mm},
	left=5mm,enhanced,
  } 

\DeclareTCBInputListing[]{\mylistingfs}{ m }{% m é um argumento mandatório (veja a documentação do pacote xparse)
  colback=blue!5!white,colupper=InpePreto,coltitle=black,colframe=InpeLaranja!90,listing only,
  title={\hspace{-5mm}#1},
  fonttitle=\small\sffamily\bfseries,
  arc=0pt, outer arc=0pt, box align=center, halign=center, valign=center, center,
  width=0.99\paperwidth, height=0.99\paperheight,
  listing engine=minted,
	minted style=colorful,
  minted language=latex,
	minted options={fontsize=\footnotesize,breaklines,autogobble,linenos,numbersep=3mm},
	left=5mm,enhanced,
  } 

% Sentenças individuais Loren Lipsum (pacote lipsum)
% REF: https://tex.stackexchange.com/questions/254901/one-sentence-of-dummy-text
% store a big set of sentences
\unpacklipsum[1-100] % it was \UnpackLipsum before version 2.0
\ExplSyntaxOn
% unpack \lipsumexp
\seq_new:N \g_lipsum_sentences_seq
\cs_generate_variant:Nn \seq_set_split:Nnn { NnV }
\seq_gset_split:NnV \g_lipsum_sentences_seq {.~} \lipsumexp

\NewDocumentCommand{\lipsumsentence}{>{\SplitArgument{1}{-}}O{1-7}}
 {
  \lipsumsentenceaux #1
 }
\NewDocumentCommand{\lipsumsentenceaux}{mm}
 {
  \IfNoValueTF { #2 }
   {
    \seq_item:Nn \g_lipsum_sentences_seq { #1 }.~
   }
   {
    \int_step_inline:nnnn { #1 } { 1 } { #2 }
     {
      \seq_item:Nn \g_lipsum_sentences_seq { ##1 }.~
     }
   }
 }
\ExplSyntaxOff
